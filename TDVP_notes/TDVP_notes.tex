\documentclass[prb]{revtex4}
%\documentclass[a4paper,10pt]{article}


%%%%%%%Specific packages%%%%%%%%%%%%%
\usepackage[usenames,dvipsnames]{color}

\usepackage{pdfpages}
\usepackage[british]{babel} 
%\usepackage[svgnames,x11names]{xcolor}
%\usepackage{graphics}
\usepackage{mathtools}
\usepackage{braket}

\usepackage[b]{esvect} % Vector notation 
\usepackage{empheq}    % boxed equation 

\usepackage{amssymb}
\usepackage{amsfonts}
\usepackage{amsmath}
\usepackage{graphicx}
\usepackage{textcomp}   % text companion fonts
%\usepackage{dcolumn}
\usepackage{bm}         % bold math
\usepackage{microtype}  % makes pdf looks better
%\usepackage{lmodern}
%\fontfamily{garamond}
%\fontfamily{\familydefault}
%\renewcommand*\familydefault{\sfdefault} %% Only if the base font of the document is to be sans serif
%\usepackage[sf]{titlesec}

%\usepackage{mathrsfs}
%\usepackage{bbold}
%\usepackage{epsfig}
%\usepackage{booktabs}    % table format 
%\usepackage{wasysym}
%\usepackage{pmat}       % also matrix partition 
%\usepackage{fancybox} 
%\usepackage{subfig}      % figure side by side
%\usepackage{sidecap}     % side caption
%\usepackage{setspace}    % for line spacing 
%\usepackage{ctable}      % better table spacing
%\usepackage[OT1]{fontenc} % font 
%\usepackage{helvet}      % Helvetica
\usepackage{mathptmx}    % Times 
%\usepackage[sc,osf]{mathpazo}
%\usepackage[version=3]{mhchem} % chemical formulas 
%\usepackage{showkeys}    % switch it off when not needed.
%\usepackage{flafter}    % place figures/tables after references in text 
%\usepackage[left=1.5in, right=1in, top=1in, bottom=1in, includefoot,
 %                    headheight=13.6pt]{geometry}   % changing the page layout  
%\usepackage[latin1]{inputenc}
%\usepackage{euler}       % Font for math mode 
%\setlength{\abovecaptionskip}{5pt}
\usepackage[breaklinks]{hyperref} %%% Hyper-linking
\hypersetup{colorlinks=false}

%%% paragraph setting
\setlength{\parskip}{2.0ex plus 0.2ex minus 0.2ex}
\setlength{\parindent}{0pt}
\renewcommand{\arraystretch}{2.0}


%%% hack boxed 
\newcommand*{\boxedcolor}{Lavender}

\definecolor{mycol}{RGB}{255, 229, 255}
\newcommand*\mybox[1]{\colorbox{mycol}{\hspace{1em}#1\hspace{1em}}}

\makeatletter
\renewcommand{\boxed}[1]{\textcolor{\boxedcolor}{%
\fbox{\normalcolor\m@th$\displaystyle#1$}}}
\makeatother

%%% Few specific commands  %%%
\newcommand{\eq}[1]{\begin{align}#1\end{align}}

\newcommand{\ba}{\begin{array}}
\newcommand{\ea}{\end{array}}

\newcommand{\bit}{\begin{itemize}}
\newcommand{\eit}{\end{itemize}}

\newcommand{\br}{{\bf r}}
\newcommand{\Le}{\left}
\newcommand{\Ri}{\right}
\newcommand{\nn}{\nonumber}
\newcommand{\R}{\rho}
\newcommand{\f}{\frac}
\newcommand{\bs}{\boldsymbol}
\newcommand{\B} {\bf}
\newcommand{\mbf}{\mathbf}
\newcommand{\mrm}{\mathrm}
\newcommand{\tr}{\textrm}
\newcommand{\tbl}{\textcolor{blue}}
\newcommand{\trd}{\textcolor{red}}
\newcommand{\mc}{\mathcal}
\newcommand{\mf}{\mathfrak}
\newcommand{\dg}{\dagger}
\newcommand{\om}{\omega}
\newcommand{\ra}{\rangle}
\newcommand{\la}{\langle}
\newcommand{\ua}{\uparrow}
\newcommand{\da}{\downarrow}

\newcommand{\ii}{i}
\newcommand*\conj[1]{{#1^\star}}
\newcommand*\conjk[1]{{#1^{k\star}}}
\newcommand*\conjp[1]{{#1^{p\star}}}
\newcommand*\dotconjp[1]{{\dot{#1}^{p\star}}}
\newcommand*\dotp[1]{{\dot{#1}^{p}}}
\newcommand*\p[1]{{\dot{#1}^{p}}}
\newcommand*\kp{\boldsymbol{\kappa}}
\newcommand{\wk}{\omega_k}

\newcommand{\ff}[2]{\la f_{#1} | f_{#2} \ra}
\newcommand{\fg}[2]{\la f_{#1} | h_{#2} \ra}
\newcommand{\gf}[2]{\la h_{#1} | f_{#2} \ra}
\newcommand{\ggnm}[2]{\la h_{#1} | h_{#2} \ra}

%\newcommand{\fm}{f_m}
%\newcommand{\pj}{p_j}
%\newcommand{\pm}{p_m}

% begin document
\begin{document}
\title{Dynamical equations of a qubit coupled to a cavity decaying into a bosonic bath -- via SPIN-BOSON}
\author{Nicolas Gheeraert}
\date{\today}
\maketitle
%%%%%%%%%%%%%%%%%%%%%%%%%%%%%%%%%%%%%%%%%%%%%%%%%%%%%%%%%%%%%%%%


\section{General algorithm}


%
We start with the following wavefunction
\eq{
|\Psi\ra = \sum_{i}^{N_q-1} \sum_{n}^{\rm ncs} p_{i,n} \ket{i} \ket{z_{i,n}}
}
Here $p_{i,n}$ and $z_{i,n}^p$ are all complex and time dependent variational parameters. 

The Lagrangian is given by:

\eq{
\mc{L}  = \braket{ \Psi | \frac{i}{2}  \overleftrightarrow{\partial_t} - \hat H | \Psi  }
}

Explicitely:

\eq{
\braket{\Psi|  \vv*{\partial}{t} | \Psi } 
&=  \left(\sum_{m} p_m^\star  \bra{z_m} \right) \vv*{\partial}{t} \left(\sum_{n} p_n \ket{z_n} \right)  \notag \\
&= \sum_{mn} p_m^\star  \braket{z_m|z_n} \biggl( \dot{p}_n -\frac{1}{2}p_n \Bigl( \sum_p \dot{z}_n^p z_n^{p\star} + z_n^p \dot{z}_n^{p\star} - 2 z_m^{p\star} \dot{z}_n^p \Bigr)  \biggr)
}
where we have used: 
\eq{
\la z_n | \vv*{\partial}{t} | z_m \ra &= -\f{1}{2} \Le( \sum_p  \dot{z}_m^p z_m^{p\star} +
z_m^p \dot{z}_m^{p\star} -2 z_n^{p\star}\dot{z}_m^p \Ri) \la z_n | z_m \ra \nn 
}
Since we have that:
\eq{
\braket{\Psi|   \overleftarrow{\partial_t} | \Psi }  =  \braket{\Psi| \overrightarrow{\partial_t} | \Psi } ^\star,
}
we obtain:
\eq{
\mc{L}  &= \frac{i}{2}\sum_{mn}  \braket{z_m|z_n} \biggl[ p_m^\star \dot{p}_n - p_n \dot{p}_m^\star - \frac{1}{2}p_m^\star p_n \Bigl( \sum_p \dot{z}_n^p z_n^{p\star} + z_n^p \dot{z}_n^{p\star} - 2 z_m^{p\star} \dot{z}_n^p - \dot{z}_m^{p\star} z_m ^p- z_m^{p\star} \dot{z}_m^p + 2 z_n^p \dot{z}_m^{p\star} \Bigr)  \biggr] -  \braket{ \Psi |  \hat H | \Psi  }
}

The Euler-Lagrange equations are: 
\eq{
\f{d}{d t} \f{\partial \mc{L}}{\partial \conj{\dot{p}_j}} - \f{\partial
\mc{L}}{\partial \conj{p_j}} =0  \quad \text{and} \qquad
\f{d}{d t} \f{\partial
\mc{L}}{\partial \conjp{\dot{z}_j}} - \f{\partial
\mc{L}}{\partial \conjp{z_j}} =0.
}

After $\conj{p_j}$ variation we get
\begin{empheq}[box=\fbox]{equation}
 \sum_m \Le( \dot{p}_m - \frac{1}{2}p_m \kp_{mj} \Ri) M_{jm} = -\ii \f{\partial E}{\partial \conj{p_j}}  \equiv P_j
\label{equation1}
\end{empheq}

After $\conjp{z_j}$ variation we get
\eq{
  \sum_m p_m \conj{p_j} \dot{z}_m^p  M_{jm}
-\f{1}{4} \sum_m \Le(2 \dot{p}_m - p_m \kp_{mj}  \Ri) \conj{p_j} (z_j^p-2z_m^p)
M_{jm} 
+\f{1}{4} \sum_m \Le(2 \conj{\dot{p}_m} - \conj{p_m}\conj{\kp_{mj}} \Ri) p_j
z_j^p M_{mj} = -i \f{\partial E}{\partial \conjp{z_j}}
\label{rawequation2_B}
}
where we have defined:
\eq{
&M_{jm} = \braket{z_j|z_m} \\
&\kappa_{mj} =  \sum_p \dot{z}_m^p z_m^{p\star} + \dot{z}_m^{p\star} z_m^p - 2 z_j^{p\star} \dot{z}_m^p
}
Using (\ref{equation1}) to simplify (\ref{rawequation2_B}), we get:
\begin{empheq}[box=\fbox]{equation}
\sum_m p_m  \dot{z}_m^p M_{jm}  + \sum_m ( \dot{p}_m
- \f{1}{2} p_m \kp_{mj})  z_m^p M_{jm}  =  Z_j^p,
\label{equation2_Z}
\end{empheq}
where we have defined:
\eq{
Z_j^p = -\ii \Le[\f{\partial E}{\partial \conjp{z_j}}\frac{1}{p_j^\star}  + \f{1}{2} \Le( \f{\partial E}{\partial p_j^\star}
 + \f{\partial E}{\partial p_j } \frac{p_j}{p_j^\star}  \Ri) z_j^p \Ri] 
}

%%%%%%%%%%%%%%%%%%%%%%%%%%%%%%%%%%%%%%%%%%%%%%%%%%%%%%%%%%%%%%%%
\bibliographystyle{unsrt}
%\bibliographystyle{abbrv}
%\bibliography{BibFiles/biblio}
%%%%%%%%%%%%%%%%%%%%%%%%%%%%%%%%%%%%%%%%%%%%%%%%%%%%%%%%%%%%%%%%
\end{document}
